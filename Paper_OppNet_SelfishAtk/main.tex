\documentclass[10pt, conference, letterpaper]{IEEEtran}
%\usepackage{cite}
%\usepackage{graphicx}
%\usepackage{amsmath}
%\usepackage{amssymb}



\usepackage{cite,comment,url}
%%\usepackage[dvips]{graphicx}
\usepackage{caption}
\usepackage[caption=false,font=footnotesize]{subfig}
\usepackage{float}
\usepackage{enumerate}
%%\usepackage{epstopdf}

\usepackage{algorithmic}
\usepackage{algorithm}
\usepackage{multirow,subfig,epsfig}
\usepackage{amstext}


\newtheorem{mydef}{Definition}
\newtheorem{mytheo}{Theorem}

\usepackage{graphicx}
%%\usepackage{epstopdf}

\usepackage[namelimits]{amsmath}
\usepackage{amssymb}
\usepackage{cite}
\usepackage{array}
\usepackage{mdwmath}
\usepackage{mdwtab}

\usepackage{subfig}

\usepackage{amsmath,amsthm}
\newtheorem{lem}{Lemma}
%%\usepackage{subcaption}

%% for highlight 2019-06-03
\usepackage{color, soul}

\setlength{\abovedisplayskip}{1mm}
\setlength{\belowdisplayskip}{1mm}

\IEEEoverridecommandlockouts

\begin{document}

\bibliographystyle{IEEEtran}

\title{
OppNet
}
\author{
A, B, C, D\\
Key Lab of CNII, MOE, Southeast University, Nanjing, P.R. China\\
Email: \{XXX, XXX\}@seu.edu.cn
}
\maketitle

\section{Related Work}
\label{sec:related}
Data dissemination in OppNets has attracted many attentions
to improve its performance.

However, these nodes in the OppNet belong to the individuals,
which always try to maximise self-benefit and ignore the network benefit.
Due to the individuals' selfish behaviors, 
including dropping other nodes' messages, spamming its messages and colluding with other malicious nodes,
the forwarding performance in OppNet will be degraded.
To alleviate this situation,
the researchers have done many works.
Some of them thought that 
incentive the nodes' cooperation to solve the selfish attack.


\bibliographystyle{ieeetr}
\bibliography{reference}


%\end{CJK}
\end{document}


